%%%%%%%%%%%%%%%%%%%%%%%%%%%%%%%%%%%%%%%%%
% Short Sectioned Assignment
% LaTeX Template
% Version 1.0 (5/5/12)
%
% This template has been downloaded from:
% http://www.LaTeXTemplates.com
%
% Original author:
% Frits Wenneker (http://www.howtotex.com)
%
% License:
% CC BY-NC-SA 3.0 (http://creativecommons.org/licenses/by-nc-sa/3.0/)
%
%%%%%%%%%%%%%%%%%%%%%%%%%%%%%%%%%%%%%%%%%

%----------------------------------------------------------------------------------------
%   PACKAGES AND OTHER DOCUMENT CONFIGURATIONS
%----------------------------------------------------------------------------------------

\documentclass[paper=a4, fontsize=11pt]{scrartcl} % A4 paper and 11pt font size

\usepackage[T1]{fontenc} % Use 8-bit encoding that has 256 glyphs
\usepackage{fourier} % Use the Adobe Utopia font for the document - comment this line to return to the LaTeX default
\usepackage[english]{babel} % English language/hyphenation
\usepackage{amsmath,amsfonts,amsthm} % Math packages

\usepackage{lipsum} % Used for inserting dummy 'Lorem ipsum' text into the template

\usepackage{graphicx} % Required for including pictures
\usepackage{wrapfig}
\usepackage{float}

\usepackage[top=.9in, bottom=.9in, left=1in, right=1in]{geometry}

\usepackage{sectsty} % Allows customizing section commands
\allsectionsfont{\normalfont\scshape} % Make all sections centered, the default font and small caps

\usepackage{fancyhdr} % Custom headers and footers
\pagestyle{fancyplain} % Makes all pages in the document conform to the custom headers and footers
\fancyhead{} % No page header - if you want one, create it in the same way as the footers below
\fancyfoot[L]{} % Empty left footer
\fancyfoot[C]{} % Empty center footer
\fancyfoot[R]{\thepage} % Page numbering for right footer
\renewcommand{\headrulewidth}{0pt} % Remove header underlines
\renewcommand{\footrulewidth}{0pt} % Remove footer underlines
\setlength{\headheight}{0pt} % Customize the height of the header

\numberwithin{equation}{section} % Number equations within sections (i.e. 1.1, 1.2, 2.1, 2.2 instead of 1, 2, 3, 4)
\numberwithin{figure}{section} % Number figures within sections (i.e. 1.1, 1.2, 2.1, 2.2 instead of 1, 2, 3, 4)
\numberwithin{table}{section} % Number tables within sections (i.e. 1.1, 1.2, 2.1, 2.2 instead of 1, 2, 3, 4)

\setlength\parindent{0pt} % Removes all indentation from paragraphs - comment this line for an assignment with lots of text

\graphicspath{{./graphics/}} % Specifies the directory where pictures are stored

%----------------------------------------------------------------------------------------
%   TITLE SECTION
%----------------------------------------------------------------------------------------

\newcommand{\horrule}[1]{\rule{\linewidth}{#1}} % Create horizontal rule command with 1 argument of height

\title{ 
\normalfont \normalsize 
\textsc{6.885 From ASCII to Answers} % Your university, school and/or department name(s)
\horrule{0.5pt} % Thin top horizontal rule
\large The Librarian: Entity Matching Across Media Types % The assignment title
\horrule{1pt} % Thick bottom horizontal rule
}

\author{Joshua Blum, Nolan Eastin \\ \{joshblum, neastin\}@mit.edu}

\date{\normalsize\today} % Today's date or a custom date
\begin{document}

\maketitle % Print the title

\section{Introduction and Motivation}
%TODO
%------------------------------------------------
Entity matching is nontrivial and especially difficult when each entity is large. The Librarian aims to perform automatic entity resolution across various media types. Given sets of text, audio, image, and video files, The Librarian can merge the sets of entities by de-­duplicating the files, gathering metadata from trusted sources, (i.e. IMDB, Rotten Tomato, and Spotify APIs), and categorizing the media based on the metadata that is found. \\

The motivation behind the system is to be able to handle large dumps of data as well as incremental updates over a large corpus of files with many contributers. The Librarian serves as a background system upon which clients can be built which handler user interaction to add and modify data within the system.\\

The initial system has been tested with approximately 6 TB of seed data that has been collected from several sources. This data was used as training data for testing and developing our matching algorithms. Only movie entities were categorized from this corpus, although the matching algorithms can be extended to work with other media types such as audio, image, or text files.

In addition to trying to algorithmically resolve two entities, crowd sourcing is used to establish ground truth where necessary. If there are multiple matches for a single entity a user can select the correct metadata for the entity or input their own.\\

We begin by discussing related work (\ref{sec:related-work}), providing a system design overview (\ref{sec:system-overview}), showing an analysis of the system's performance (\ref{sec:performance}), list goals for future work (\ref{sec:future-work}), and conclusions of the project (\ref{sec:conclusions}).

\section{Related Work}
\label{sec:related-work}
%TODO NOLANDDDDD
%------------------------------------------------
Audio Fingerprinting: 
\section{System Overview}
\label{sec:system-overview}


%------------------------------------------------

\subsection{Architecture}
\label{sec:architecture}
%TODO

%------------------------------------------------

\subsection{Implementation}
\label{sec:implementation}
%TODO

%------------------------------------------------
\subsubsection{Handlers}
\label{sec:handlers}
%TODO

%------------------------------------------------
\subsubsection{Identifiers}
\label{sec:identifiers}
%TODO

%------------------------------------------------

\section{Performance}
\label{sec:performance}
%TODO

%------------------------------------------------

\section{Future Work}
\label{sec:future-work}
%TODO

%------------------------------------------------
\section{Conclusions}
\label{sec:conclusions}
%TODO

%------------------------------------------------


\end{document}`'